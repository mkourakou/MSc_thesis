\chapter{Introduction}
\label{chp:intro}
\captionsetup{width=0.75\textwidth}

The first chapter is an introductory one and its scope is to explain the motivation that stands behind the work of the present thesis and map the steps followed in the next chapters.

\section{Motivation}

Regular matter in the Universe interacts with light, which allows it to convert potential or kinetic energy into radiation, lose energy by emitting light and condense into galaxies, stars and planets. The study of galaxy formation and evolution can help us understand the physics of gas and its conversion to stars.\\
In the local Universe, star formation in most galaxies is very inefficient. Large amounts of energy are required to counteract the growth that would otherwise naturally arise from a galaxy's gravitational pull, the primary sources of this energy (feedback) might be related to the accretion process and are a focus of research in extragalactic astrophysics.\\
Radio galaxies host relativistic jets in their core, produced by the accretion process and appear to be quiescent in the Local Universe\cite{Best2005}, while the cosmic time at which their stellar mass was formed or the quenching timescale have not been explored at large. Additionally, Radio galaxies with molecular gas content (which is the fuel of star formation) are very rare in the Local Universe. Retrieving information that can be linked to star formation in this type of feedback-ridden galaxies across cosmic time would be valuable for understanding star formation and quenching in the Universe. \\ \\
The estimation of stellar mass content for a sample of gas rich radiogalaxies, their stellar mass function and its evolution in comparison with star-forming galaxies will be the central theme of the present thesis.

\section{Thesis Outline}
The following chapters of this thesis are structured as follows:\\ \\
Chapter 2: Statistical properties of galaxies are reviewed, in particular the star formation rate and galaxy stellar mass relation along with the galaxy stellar mass function. The Star Forming Main Sequence is defined, its importance as an evolutionary threshold is highlighted.\\ \\
Chapter 3: The ARC survey is concisely presented and the photometric data integration and reduction is described for the working dataset of radiogalaxies.\\ \\
Chapter 4: An overview of Spectral Energy Distributions of galaxies, their data assortment and information is presented. The basic structure and components of a radio loud galaxy are described. The approach after which they are modelled is described along with the radiative processes that justify and establish it. State-of the-art models are chosen and their full parametrisation is clarified.  \\ \\
Chapter 5: Bayesian inference principles are briefly revised. The Bayesian SED fitting algorithm developed for the present work and adapted for the ARC sample of galaxies is described, including the SED model implementation, sampling method, likelihood and prior functions. The algorithm is tested in parts and as a whole on synthetic data and proceeded to fit the models to the working sample of galaxies' SEDs. The sampled posterior and stellar mass histograms are plotted along with 45 radiogalaxies' fit SEDs.\\ \\
Chapter 6: The inferred parametres along with archival properties are manipulated in order to produce stellar mass function estimates and gas mass-to-stellar mass relations. Methods followed, techniques and assumptions are described in detail.\\ \\ 
Chapter 7: The results are compared with literature, broadly commented and conclusions are drawn. Possible improvements both on the computational implementation and the method are remarked.\\ \\ 
In the present work the \code{python 3} programming language has been used for data manipulation, analysis and visualisation, implementing \code{numpy}\cite{numpy}, \code{pandas}\cite{pandas}, \code{scipy}\cite{scipy}, \code{astropy}\cite{astropy}, \code{matplotlib}\cite{matplotlib} and \code{UltraNest}\cite{ultranest2021} packages.










