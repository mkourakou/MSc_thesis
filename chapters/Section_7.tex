\clearpage
\section{Section Two}
\label{sec:Section_Name}
...
\subsection{Exemplary Figure}
\label{subsec:Section_Name/fig}
...
\begin{figure}[htbp]
    \centering
    \includegraphics[width=.5\linewidth]{./Figures/UoC_Logo.png}
    \caption{Exemplary Figure}
    \label{fig:UoC}
\end{figure}


\subsection{Exemplary Figure Referencing}
\label{subsec:Section_Name/fig_rfs}

See Figure \ref{fig:UoC} for details. Additional information can be
found in the footnote \footnote{Image taken from \url{https://en.wikipedia.org/wiki/File:Siegel_Uni-Koeln_(Grau).svg}.}.





%------------------------------------------------------


\chapter{Results}
\label{chp:Res}
\captionsetup{width=0.75\textwidth}
%\clearpage

\section{ARC Stellar Mass Function} \label{subsec:Res/ARC_SMF}
There are two main ways to estimate the stellar mass function for a galaxy sample\cite{Leja2020}, the first one is the $1/V_{\mbox{\tiny{max}}}$ method, introduced by Schmidt (1968)\cite{Schm1968} and refined by Avni \& Bahcall (1980)\cite{Avni1980} which calculates the number density of galaxies in bins of stellar mass with 
\begin{equation}
    \Phi_j = \dfrac{1}{\Delta \log M_{\star j}} \sum_{i=1}^N \, \dfrac{1}{V_{\mbox{\tiny{max}}}} \label{eq:SMF_schm} 
\end{equation} 
where $\Phi_j$ is the number density of galaxies in bin $j$, $\Delta \log M_{\star j}$ is the width in logarithmic scale of bin $j$, $N$ is the number of sources inside the bin and $V_{\mbox{\tiny{max}}}$ is the maximum comoving volume out to which these objects could be detected at the survey's limit. 
The second approach to estimate the stellar mass function is introduced by Sandage et al. (1979)\cite{Sand1979} and is a parametric maximum likelihood estimator that typically assumes the functional form of a Schechter\cite{Sche1976} function. 

%----------------------------------------------------------------------------------%
\subsection{Maximum Volume} \label{subsec:Res/V_max}
\subsubsection*{Calculating the Maximum Comoving Distance for a source}
As shown in equation \ref{eq:LumiD}: $ \;\; D_{\mbox{\tiny{L}}} = (1+z) \,D_{\mbox{\tiny{com}}} $\\
And from the definition of redshift: 
\begin{equation}
\nu_{\mbox{\tiny{obs}}} = \dfrac{\nu_{\mbox{\tiny{emitted, restframe}}}}{1+z_{\mbox{\tiny{source}}}}
\label{eq:redshift_freq}
\end{equation}
Virtually redshifting an isotropic source to $z_{\mbox{\tiny{limit}}}$, the observed photons at frequency $\nu_{\mbox{\tiny{limit}}} = 1.4\mbox{GHz}$ were emitted at frequency $\nu_{\mbox{\tiny{limit}}} = (1+z_{\mbox{\tiny{limit}}})\times1.4\mbox{GHz}$, the same holds for the observed redshift of the source, thus:
\begin{equation*} \begin{cases}
L_{\mbox{\tiny{bolom}}}\Big|_{\mbox{\tiny{observed}}}  = L_\nu \big( (1+z_{\mbox{\tiny{source}}})1.4\mbox{GHz} \big)\times (1+z_{\mbox{\tiny{source}}})1.4\mbox{GHz} \\ \\
L_{\mbox{\tiny{bolom}}}\Big|_{\mbox{\tiny{limit}}}  = L_\nu \big( (1+z_{\mbox{\tiny{limit}}})1.4\mbox{GHz} \big)\times (1+z_{\mbox{\tiny{limit}}})1.4\mbox{GHz}
\end{cases} \xRightarrow[]{\ref{eq:Flux-LumiD}}
\end{equation*}  
\begin{equation*} \xRightarrow[]{\ref{eq:Flux-LumiD}} \begin{cases}
L_{\mbox{\tiny{bolom}}}\Big|_{\mbox{\tiny{observed}}}  = f_{\mbox{\tiny{ν, obs}}}(1.4\mbox{GHz}) \times (1.4\mbox{GHz})  \times 4\pi \times D^2_{\mbox{\tiny{L}}} (z_{\mbox{\tiny{source}}} ) \\ \\
L_{\mbox{\tiny{bolom}}}\Big|_{\mbox{\tiny{limit}}} = f_{\mbox{\tiny{ν, limit}}}(1.4\mbox{GHz}) \times (1.4\mbox{GHz})  \times 4\pi \times D^2_{\mbox{\tiny{L}}} (z_{\mbox{\tiny{limit}}} )
\end{cases} 
\end{equation*} 
And since the bolometric luminosity of the source on its current redshift should be the same as its bolometric luminosity on the limit redshift (since there is no Doppler redshift from relativistic motion), the following is an identity:
\begin{equation*}
L_{\mbox{\tiny{bolom}}}\Big|_{\mbox{\tiny{observed}}}   = L_{\mbox{\tiny{bolom}}}\Big|_{\mbox{\tiny{limit}}} \xRightarrow[]{}  \end{equation*} 
\begin{equation*}
f_{\mbox{\tiny{ν, obs}}}(1.4\mbox{GHz}) \times D^2_{\mbox{\tiny{L}}} (z_{\mbox{\tiny{source}}} )  = f_{\mbox{\tiny{ν, limit}}}(1.4\mbox{GHz}) \times D^2_{\mbox{\tiny{L}}} (z_{\mbox{\tiny{limit}}} ) \xRightarrow[]{\ref{eq:LumiD}}  \end{equation*} 
\begin{equation} \xRightarrow[]{\ref{eq:Flux-LumiD}} (1+ z_{\mbox{\tiny{limit}}})^2 \times D^2_{\mbox{\tiny{com}}} (z_{\mbox{\tiny{limit}}} ) = \dfrac{f_{\mbox{\tiny{ν, obs}}}(1.4\mbox{GHz})}{f_{\mbox{\tiny{ν, limit}}}(1.4\mbox{GHz})} \times (1+ z_{\mbox{\tiny{source}}})^2 \times D^2_{\mbox{\tiny{com}}} (z_{\mbox{\tiny{source}}} )
\label{eq:Limit_f_D}
\end{equation}
In equation \ref{eq:Limit_f_D}, the right-hand-side is comprised entirely by known quantities:
\begin{itemize}
    \item $z_{\mbox{\tiny{source}}}$ is the spectroscopic redshift of each galaxy
    \item $f_{\mbox{\tiny{ν, obs}}}(1.4\mbox{GHz})$ is the observed flux at $1.4$GHz of each galaxy (bound to be greater or equal to $0.4$Jy, as a selection criterion of the ARC survey)
    \item $f_{\mbox{\tiny{ν, limit}}}(1.4\mbox{GHz}) $ is the limit flux at $1.4$GHz, which is equal to $0.4$Jy, for a galaxy to be marginally part of the ARC survey.\\
\end{itemize}
Solving the equation \ref{eq:Limit_f_D} for $\,z_{\mbox{\tiny{limit}}} $, it is trivial to calculate the limit comoving distance $ D_{\mbox{\tiny{com}}} (z_{\mbox{\tiny{limit}}} )$.

%---------------------------------------------------------------------------------%
\subsubsection*{Calculating the Maximum Volume for a source}
As discussed in section \ref{sec:Cosmo-ComovD}, the transverse comoving distance for a flat Universe is equal to the radial one $ D_{\mbox{\tiny{com, transv}}}  = D_{\mbox{\tiny{com}}}$.\\ 
The total volume of the solid angle layer spanned when virtually moving a galaxy from $z_{\mbox{\tiny{source}}}$ to $z_{\mbox{\tiny{limit}}}$ is 
\begin{equation}
\begin{aligned}
    V_{\mbox{\tiny{s}} \longrightarrow \mbox{\tiny{lim}}} &= V_{\mbox{\tiny{lim}}} - V_{\mbox{\tiny{s}}}=  \\ 
    & = \dfrac{4\pi}{3} \times \Big[ D^3_{\mbox{\tiny{com}}} (z_{\mbox{\tiny{limit}}} ) -  D^3_{\mbox{\tiny{com}}} (z_{\mbox{\tiny{source}}} )] 
\end{aligned} 
\label{eq:Volume_tot}
\end{equation}
The effective FOV of an ARC galaxy spans a solid angle $\Omega_{\mbox{\tiny{eff, ARC}}} = 75 \mbox{arcmin}^2 \approx 6.3462 \times 10^{-06}\; \mbox{steradian}$ (as discussed in Section \ref{sec:ARCsurv}), thus the maximum comoving volume spanned by the FOV of an ARC galaxy is
\begin{equation}
\begin{aligned}
    V_{\mbox{\tiny{max}}}  & = \dfrac{\Omega_{\mbox{\tiny{eff, ARC}}} }{4\pi}  \times \dfrac{4\pi}{3} \times \Big[ D^3_{\mbox{\tiny{com}}} (z_{\mbox{\tiny{limit}}} ) -  D^3_{\mbox{\tiny{com}}} (z_{\mbox{\tiny{source}}} )\Big] =\\
    & = \dfrac{\Omega_{\mbox{\tiny{eff, ARC}}} }{3} \times \Big[ D^3_{\mbox{\tiny{com}}} (z_{\mbox{\tiny{limit}}} ) -  D^3_{\mbox{\tiny{com}}} (z_{\mbox{\tiny{source}}} )\Big] =\\
    & = \dfrac{6.3462 \times 10^{-06}}{3} \times \Big[ D^3_{\mbox{\tiny{com}}} (z_{\mbox{\tiny{limit}}} ) -  D^3_{\mbox{\tiny{com}}} (z_{\mbox{\tiny{source}}} )\Big]
\end{aligned} 
\label{eq:Volume_max}
\end{equation}
%As discussed in Section \ref{sec:ARCsurv}, the ARC survey covers the sky north of $-40^o$ and south of  ($82\%$ of the celestial sphere)
%$\Omega_{\mbox{\tiny{eff, ARC}}} = 75 \mbox{arcmin}^2 \approx 6.3462 \times 10^{-06}\; \mbox{steradian}$




%---------------------------------------------------------------------------------%
\subsection{Completeness Correction} \label{subsec:Res/ComplCorr}
\begin{figure}
    \centering
    \includegraphics[width=.7\linewidth]{figures/AudibertCompleteness.png}
    \caption{The completeness correction so that by multiplying the final ARC source sample in different redshift intervals can reproduces the number of sources in the flux-limited NVSS. Plot lifted from Audibert et al. (2022)\cite{Audibert2022} for 66 ARC sources}
    \label{fig:ComplPlot}
\end{figure}
In order to estimate the stellar mass function the $1/V_{\mbox{\tiny{max}}}$ method is addopted here, and for the purposes of the ARC survey (which is representative of the NVSS survey) the equation \ref{eq:SMF_schm} is modified to 
\begin{equation}
\Phi_j = \dfrac{1}{\Delta \log M_{\star j}} \sum_{i=1}^N \, \dfrac{\omega(\delta z) }{V_{\mbox{\tiny{max}}}} \label{eq:SMF_arc}
\end{equation}
where $\omega (\delta z)$ is a weight corresponding to the completeness correction with respect to the NVSS survey addopted by Audibert et al. (2022)\cite{Audibert2022} and shown in Figure \ref{fig:ComplPlot}, that is 
\begin{equation}
\omega(\delta z) = \dfrac{\mbox{\small{Number of NVSS sources with spectroz in δz slice per sterad}}}{\mbox{\small{Number of ARC sources with spectroz in δz slice per sterad}}}=  \dfrac{\dfrac{N_{\mbox{\tiny{NVSS}}}(\delta z)}{\Omega_{\mbox{\tiny{tot, NVSS}}}}} { \dfrac{N_{\mbox{\tiny{ARC}}}(\delta z)}{\Omega_{\mbox{\tiny{tot,ARC}}}} } 
\label{eq:weights_arc}
\end{equation}
where $\Omega_{\mbox{\tiny{tot, ARC}}}$ is the ARC survey's coverage of the celestial sphere's solid angle, $\Omega_{\mbox{\tiny{tot, NVSS}}}$ is the NVSS survey's coverage of the celestial sphere, $N_{\mbox{\tiny{ARC}}}(\delta z)$ is the total number of ARC sources in a redshift region  $\delta z$ and $N_{\mbox{\tiny{NVSS}}}(\delta z)$ is the number of NVSS sources with spectroscopically confirmed redshift in the same redshift region.\\
As shown in equation \ref{eq:OmegaTotARC}, the ARC survey covers a solid angle of $\Omega_{\mbox{\tiny{tot, ARC}}} = 3.94\pi \;\mbox{steradian}$
The NVSS survey\cite{NVSS} covers the sky north of $-40^o$, that is 
\begin{equation*}
    \Omega_{\mbox{\tiny{tot, NVSS}}} = \int_{0}^{2\pi}\,d\phi\, \int_{-2\pi/9}^{\pi}\, \cos \theta\,d\theta = 2\pi \times (1+\sin\frac{2\pi}{9}) \; \mbox{steradian}= 3.28 \pi\; \mbox{steradian}
\end{equation*}
Audibert et al. (2022)\cite{Audibert2022} calculate the total number of NVSS sources within the redshift interval $z\in[0,2.5]$ (which is the redshift range of the entire ARC sample) to be $N_{\mbox{\tiny{NVSS, tot}}} =  9331$ and in order to estimate the number of NVSS sources with spectroscopic redshift, a fraction of $f_{\mbox{\tiny{spectro z}}} = 0.06$ is assumed based on other surveys (SDSS and FIRST), leading to $N_{\mbox{\tiny{NVSS}}} = 0.06\times 9331 = 559.86$. Although the total number of ARC sources with spectroscopic redshift is 120, the present work used 45 for science, thus the completeness correction for the redshift range of $z\in[0,2.5]$ is 
\begin{equation}\begin{aligned}
    \omega(0< z<2.5) &= \dfrac{N_{\mbox{\tiny{NVSS}}}(0< z<2.5)}{\Omega_{\mbox{\tiny{tot, NVSS}}}} \times  \dfrac{\Omega_{\mbox{\tiny{tot,ARC}}}}{N_{\mbox{\tiny{ARC}}}(0< z<2.5)}  =  \\ 
    & = \dfrac{559.86}{3.28 \pi} \times  \dfrac{3.94\pi}{45} = \\ & = 14.94
    \end{aligned} 
\label{eq:weight_tot}
\end{equation}

%--------------------------------------------------------------------------------------------
\subsection{Redshift Binning} \label{subsec:Res/ComplCorr}


\begin{table}    \caption{Crucial values and limits for calculating completeness of 45 ARC galaxies} \label{tab:ARC_45complval}
\begin{tabular}{cccccccc}
    \hline  %\toprule
    Source name & $f_{\mbox{\tiny{1.4GHz, source}}}$ &  $z_{\mbox{\tiny{source}}}$ & $z_{\mbox{\tiny{limit}}}$  & $V_{\mbox{\tiny{com, source}}}$ & $V_{\mbox{\tiny{com, limit}}}$ & log$M_{\star, \mbox{\tiny{source}}}$ & log$M_{\star, \mbox{\tiny{limit}}}$ \\
    \hline   %\midrule
    \hline   %\midrule
    %---------------------------------------------%
    %-----  0  ---------%
    J0006-0623 & 1.754 & 0.347 & 0.644  & 10621412470 & 53619010921  & 11.973 & 2.731  \\
    \addlinespace      
    %-----  1  ---------%    
    J0009-3216  & 0.505 & 0.026 & 0.029  & 5451077 & 7667186  & 11.329 & 8.970  \\
    \addlinespace
    %-----  2  ---------% 
    J0048+3157 & 0.401 & 0.015 & 0.015  & 1106755 & 1110842  & 10.434 & 10.408 \\
    \addlinespace
    %-----  3  ---------%
    J0057+3021 & 1.715 & 0.017 & 0.034  & 1518505 & 12811912  & 11.554 & 2.694 \\
    \addlinespace
     %-----  5  ---------%
    J0106-4034 & 0.801 & 0.584 & 0.776  & 42025675299 & 84426729631  & 10.708 & 5.349\\
    \addlinespace
     %-----  6  ---------%
    J0112-6634 & 0.431 & 1.189 & 1.226  & 218963612672 & 233195025541  & 12.224 & 11.333 \\
    \addlinespace
     %-----  8   ---------%
     J0119+3210 & 2.502 & 0.059 & 0.140  & 65948906 & 826594905  & 11.074 & 1.771 \\
    \addlinespace
     %-----  9  ---------%
     J0125-0005  & 1.467 & 1.076 & 1.821  & 177158540511 & 495526195715  & 11.933 & 3.255  \\
    \addlinespace
     %-----  16 (cont 2)  ---------%
     J0327-2239  & 0.459 & 1.898 & 2.007  & 532781704065 & 585503176478  & 11.196 & 9.759 \\
    \addlinespace
     %-----  19  ---------%
     J0504-1014 & 1.461 & 0.040 & 0.075  & 21020608 & 132875955  & 10.890 & 2.982\\
    \addlinespace
     %-----  25  ---------%
     J0758+3747& 1.734 & 0.041 & 0.083  & 21772604 & 174619443  & 11.883 & 2.742 \\
    \addlinespace
     %-----  28  ---------%
     J0840+2949 & 0.514 & 0.065 & 0.073  & 85292621 & 121422757  & 11.041 & 8.591 \\
    \addlinespace
     %-----  30  ---------%
      J0914+0245 & 0.576 & 0.427 & 0.497  & 18591860833 & 27803371016  & 11.947 & 8.295  \\
    \addlinespace
     %-----  35  ---------%
      J1000-3139 & 0.534 & 0.009 & 0.010  & 222904 & 342648  & 10.333 & 7.737 \\
    \addlinespace 
     %-----  36  ---------%
      J1008+0029 & 0.434 & 0.098 & 0.102  & 286702369 & 320847780  & 11.135 & 10.259 \\
    \addlinespace
     %-----  51  ---------%
      J1248-4118 & 3.922 & 0.010 & 0.030  & 313986 & 9075331  & 11.011 & 1.123  \\
    \addlinespace
     %-----  52  ---------%
      J1301-3226 & 1.227 & 0.017 & 0.030  & 1609884 & 8337450  & 11.001 & 3.586 \\
    \addlinespace 
     %-----  55  ---------%
      J1336-3357 & 4.372 & 0.013 & 0.040  & 637246 & 21216550  & 11.037 & 1.010  \\
    \addlinespace
     %-----  56  ---------%
      J1348+2635 & 0.700 & 0.064 & 0.084  & 84285576 & 184699036  & 11.580 & 6.617  \\
    \addlinespace
     %-----  60  ---------%
      J1407-2701 & 0.708 & 0.022 & 0.029  & 3357927 & 7742971  & 11.163 & 6.308 \\
    \addlinespace
     %-----  67  ---------%
      J1602+0157 & 8.286 & 0.105 & 0.405  & 352065728 & 16146030943  & 10.678 & 0.515  \\
    \addlinespace
     %-----  72  ---------% 
     J1945-5520 & 0.679 & 0.015 & 0.020  & 1132482 & 2470867  & 10.827 & 6.379  \\
    \addlinespace
     %-----  77  ---------% 
      J2131-3837 & 0.764 & 0.018 & 0.025  & 2057386 & 5319099  & 10.951 & 5.735  \\
    \addlinespace
     %-----  79  BREAK HERE!---------% 
      J2134-0153 & 1.880 & 1.285 & 2.413  & 256903577664 & 785233255541  & 13.141 & 2.796 \\
    \addlinespace
    %-----  82  ---------% 
     J2257-3627 & 0.840 & 0.006 & 0.009  & 71176 & 214877  & 10.781 & 5.134  \\
    \addlinespace
    %-----  83  ---------% 
     J2320+0812 & 0.627 & 0.011 & 0.014  & 478432 & 930838  & 11.090 & 7.076  \\
    \addlinespace
    %-----  85  ---------% 
     J2325-1207 & 1.627 & 0.083 & 0.159  & 177577734 & 1187583652  & 11.000 & 2.705 
 \\
    \addlinespace
    %-----  86  ---------% 
     J2341+0018 & 0.473 & 0.277 & 0.298  & 5698161687 & 6983071317  & 9.647 & 8.153  \\
    \addlinespace
    %-----  93  ---------%
      J0125-0122 & 0.826 & 0.018 & 0.026  & 1879930 & 5456446  & 10.871 & 5.263 \\
    \addlinespace
    %-----  94  ---------%
    J0156+0537 & 0.635 & 0.019 & 0.024  & 2138230 & 4217506  & 11.555 & 7.279   \\
    \addlinespace
    %-----  95  ---------%
    J0709+4836 & 0.512 & 0.019 & 0.022  & 2329418 & 3345999  & 10.993 & 8.593\\
    \addlinespace
    %-----  96  ---------%
    J2214+1350 & 2.216 & 0.026 & 0.060  & 5780173 & 68352752  & 10.598 & 1.913  \\
    \addlinespace
    %-----  103  ---------%
    J1348+2635 & 0.925 & 0.063 & 0.093  & 77671998 & 250804892  & 11.558 & 4.998 \\
    \addlinespace
    %-----  105  ---------%
     J1321+4235 & 1.687 & 0.790 & 1.415  & 88089040027 & 310812208203  & 11.593 & 2.748 \\
    \addlinespace
    %-----  108  ---------%
    J1531+2404 & 3.352 & 0.095 & 0.250  & 262964983 & 4287716737  & 11.160 & 1.332  \\
    \addlinespace
    %-----  109  ---------%
    J0747-1917 & 2.003 & 0.102 & 0.213  & 326279961 & 2739578035  & 11.841 & 2.365 \\
    \addlinespace
    %-----  111  ---------%
    J0009+1244 & 1.528 & 0.156 & 0.283  & 1118836863 & 6099733607  & 11.028 & 2.888  \\
    \addlinespace
    %-----  112  ---------%
    J011651-2052 & 3.723 & 1.410 & 3.524  & 308966929051 & 1325168544618  & 10.221 & 1.098  \\
    \addlinespace
    %-----  113  ---------%
    J0234+3134 & 0.909 & 1.575 & 2.201  & 381597436448 & 680202950647  & 11.811 & 5.199  \\
    \addlinespace
    %-----  114  ---------%
    J0408-2418 & 0.629 & 2.433 & 2.934  & 795973831784 & 1042737809694  & 11.636 & 7.398  \\
    \addlinespace
    %-----  115  ---------%
    J2106-2405 & 0.447 & 2.491 & 2.609  & 824465074464 & 882462629500  & 11.217 & 10.026 \\
    \addlinespace 
    %-----  116  ---------%
    J0242-2132 & 1.013 & 0.312 & 0.463  & 7954274949 & 23099299235  & 11.402 & 4.504  \\
    \addlinespace 
    %-----  117  ---------%
    J1305-1033 & 0.843 & 0.291 & 0.401  & 6551933406 & 15702640284  & 10.832 & 5.138  \\
    \addlinespace 
    %-----  118  ---------%
    J0403+2600 & 1.285 & 2.109 & 3.417  & 635471606586 & 1274807911997  & 12.714 & 3.958  \\
    \addlinespace
    %-----  119  ---------%
    J1347+1217 & 5.375 & 0.122 & 0.387  & 548578299 & 14305603664  & 11.011 & 0.819  \\
    \addlinespace
\end{tabular}  
\end{table}








%$\Omega_{\mbox{\tiny{tot, NVSS}}} = 0.82 \times 4\pi\; \mbox{steradian} = 3.28 \pi\; \mbox{steradian}$ 
%Audibert et al. (2022)\cite{Audibert2022} calculate the parametres for completeness correction in the redshift interval $z\in[0,2.5]$ that covers the entire ARC sample, given that the effective FOV of the ARC survey is $\Omega_{\mbox{\tiny{eff, ARC}}} = 75 \mbox{arcmin}^2 \approx 6.3462 \times 10^{-06}\; \mbox{steradian}$,  the FOV of the NVSS survey is $\Omega_{\mbox{\tiny{eff, NVSS}}} = 3.27 \pi\; \mbox{steradian}$, $N_{\mbox{\tiny{NVSS, tot}}} =  9331$ is the total number of NVSS sources with $z\in[0,2.5]$ and in order to calculate the number of NVSS sources with spectroscopic redshift, a fraction of $f_{\mbox{\tiny{spectro z}}} = 0.06$ is assumed based on other surveys (SDSS and FIRST), leading to $N_{\mbox{\tiny{NVSS}}} = 0.06\times 9331 = 559.86$ and the correction for $z\in[0,2.5]$ :
%$$ \omega_i = \dfrac{559.86}{3.27 \pi} \times \dfrac{6.3462 \times 10^{-06}}{ N_{\mbox{\tiny{ARC}}}(z\in[0,2.5])} $$








%\subsection{Exemplary Figure Referencing}\label{subsec:Section_Name/fig_rfs}
%See Figure \ref{fig:UoC} for details. Additional information can befound in the footnote \footnote{Image taken from \url{https://en.wikipedia.org/wiki/File:Siegel_Uni-Koeln_(Grau).svg}.}.













Example of \autoref{tab:1}, made using \url{https://www.tablesgenerator.com/}.


\begin{table}
    \centering
    \begin{tabular}{@{}ll@{}}
        \toprule
        \textbf{A} & \textbf{B} \\ \midrule
        1          & 2          \\
        3          & 4          \\ \bottomrule
    \end{tabular}
    \caption{Interesting results.}
    \label{tab:1}
\end{table}






