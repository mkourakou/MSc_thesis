


%, including 120 radiosources.
Galaxy observations are influenced by many physical parameters including stellar masses, star formation rates, star formation histories, metallicities, dust, black hole activity, and more. As a result, inferring accurate physical parameters requires high-dimensional models that capture this complexity. \\
%The ALMA Radio-source Catalogue (ARC) Survey is an archival radiogalaxy survey of a full spectroscopic redshift sample covering $\sim 75 \mbox{arcmin}^2$. 
The photometric Spectral Energy Distributions (SEDs) of 45 gas rich radiogalaxies from the ALMA Radio-source Catalogue (ARC) archival full spectroscopic survey at $0.06<z<2.5$ are fit and their galaxy stellar masses are assessed using a 21-parameter physical model in Bayesian inference framework. The fundamentals of Bayesian statistics, probabilistic sampling and SED modelling are explored, including an Active Galactic Nucleus (AGN) component in the SED modelling and fitting process.\\
The resulting stellar mass function of gas rich radiogalaxies in the local Universe is found to be $\sim 1$ dex lower than the stellar mass function for normal or elliptical galaxies, while in higher redshifts, the stellar mass function of gas rich radiogalaxies is found to be $\sim 4$ dex lower than the one for normal galaxies. \\
In comparison with studies regarding normal and star-forming galaxies, the gas-to-stellar mass ratio of gas rich radiogalaxies at higher redshift ($1.0<z<2.5$) is found to be closer to the one for normal galaxies, hinting that in higher redshift gas rich radiogalaxies, although uncommon, might be closer to the Star Forming Main Sequence.\\ \\
%stellar masses and ancillary parametres are presented for radiogalaxies spanning 0.06<z<2.5
%greatly improves agreement withthe evolution of the stellar mass function. We then derive a star-forming sequence that reproduces the evolutionof the mass function by design. This star-forming sequence is also well described by a broken power law, with ashallow slope at high masses and a steep slope at low masses. At z = 2, it is offset by ∼0.3 dex from the observed star-forming sequence, consistent with the mild disagreement between the cosmic star formation rate (SFR) andrecent observations of the growth of the stellar mass density. It is unclear whether this problem stems from errors instellar mass estimates, errors in SFRs, or other effects. We show that a mass-dependent slope is also seen in otherself-consistent models of galaxy evolution, including semianalytical, hydrodynamical, and abundance-matchingmodels. As part of the analysis, we demonstrate that neither mergers nor hidden low-mass quiescent galaxies arelikely to reconcile the evolution of the mass function and the star-forming sequence. 
Key words: \href{http://astrothesaurus.org/uat/594}{Galaxy Evolution} , \href{http://astrothesaurus.org/uat/2129}{Spectral Energy Distribution} , \href{http://astrothesaurus.org/uat/1882}{Astrostatistics} , \href{http://astrothesaurus.org/uat/1859}{Astronomy Data Modeling} 

%\clearpage


