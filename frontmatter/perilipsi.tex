Οι παρατηρήσεις γαλαξιών επηρεάζονται από πολλές φυσικές παραμέτρους, όπως αστρικές μάζες, ρυθμούς αστρογένεσης, ιστορικό σχηματισμού αστέρων, μεταλλικότητα, σκόνη, δραστηριότητα μελανών οπών και άλλα. Ως αποτέλεσμα, η εξαγωγή ακριβών φυσικών παραμέτρων απαιτεί πολυδιάστατα μοντέλα που αποτυπώνουν αυτή την πολυπλοκότητα.\\
Οι φωτομετρικές κατανομές φασματικής ενέργειας (SED) 45 ραδιογαλαξιών πλούσιων σε αέριο από την αρχειακή πλήρη φασματοσκοπική έρευνα του ALMA Radio-source Catalogue (ARC) σε ερυθρομετατοπίσεις $0.06<z<2.5$ προσαρμόζονται και οι αστρικές μάζες τους εκτιμώνται χρησιμοποιώντας ένα φυσικό μοντέλο 21 παραμέτρων σε πλαίσιο συμπερασματολογίας Bayes. Εξετάζονται οι βασικές αρχές της στατιστικής Bayes, της πιθανοτικής δειγματοληψίας και της μοντελοποίησης SED, περιλαμβάνοντας συνιστώσα ενεργού γαλαξιακού πυρήνα (AGN) στη διαδικασία μοντελοποίησης και προσαρμογής SED.\\
Η προκύπτουσα συνάρτηση αστρικής μάζας των πλούσιων σε αέριο ραδιογαλαξιών στο τοπικό Σύμπαν βρέθηκε να είναι $\sim 1$ dex χαμηλότερη από τη συνάρτηση αστρικής μάζας για κανονικούς ή ελλειπτικούς γαλαξίες, ενώ σε υψηλότερες ερυθρομετατοπίσεις, η συνάρτηση αστρικής μάζας των πλούσιων σε αέριο ραδιογαλαξιών βρέθηκε να είναι $\sim 4$ dex χαμηλότερη από εκείνη των κανονικών γαλαξιών. \\
Σε σύγκριση με μελέτες που αφορούν κανονικούς γαλαξίες και γαλαξίες με έντονη αστρογένεση, ο λόγος μάζας αερίου προς αστρική μάζα των πλούσιων σε αέριο ραδιογαλαξιών σε υψηλότερη ερυθρομετατόπιση ($1.0<z<2.5$) βρέθηκε να είναι πιο κοντά σε αυτόν των κανονικών γαλαξιών, υποδηλώνοντας ότι σε υψηλότερη ερυθρομετατόπιση οι πλούσιοι σε αέριο ραδιογαλαξίες, αν και σπάνιοι, είναι πιθανό να είναι πιο κοντά στην Κύρια Ακολουθία Αστρογένεσης απ' ότι στο Κοντινό Σύμπαν.\\ \\
Λέξεις-κλειδιά: \href{http://astrothesaurus.org/uat/594}{Εξέλιξη Γαλαξιών} , \href{http://astrothesaurus.org/uat/2129}{Φασματική Κατανομή Ενέργειας} , \href{http://astrothesaurus.org/uat/1882}{Αστροστατιστική} , \href{http://astrothesaurus.org/uat/1859}{Μοντελοποίηση Αστρονομικών δεδομένων} 

