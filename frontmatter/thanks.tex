%!TEX root = ../dissertation.tex
%First and foremost I would like to thank my supervisor Michael Kramer for accepting me in the “Fundamental Physics” group, for tolerating my mistakes, for always being there and, most importantly, for turning something unknown to me to an every-day experience I truly enjoy with all my heart. 
%Likewise, a simple thanks is not enough to express my gratitude to my advisors Paulo Freire, Norbert Wex & Thomas Tauris. I do not only consider them to be word- leading experts but also mentors and friends. Their ideas have influenced every line of this Thesis.
%I would like to extent my gratitude to Marten van Kerkwijk, for teaching me ev- erything I know about spectra, for openly sharing his expertise and enthusiasm, for supervising my work remotely, putting up with me and for answering all my questions no matter how trivial.

First and foremost I would like to thank my supervisor Kalliopi Dasyra for accepting me, trusting on me, tolerating my shortcomings and giving me the opportunity to work with an outstanding astrophysicist.   

I would like to thank my advisor Antonis Georgakakis for his unconditional support, for sharing his expertise, for his thoughtful input and for being an exceptional astrophysicist and an admirable mentor.

I would like to express my gratitude to Markos Polkas for guiding me through this project, openly sharing his work, answering all my questions and being a truly brilliant researcher. %lways being there, providing guidance and being 

I would like to extend my gratitude to Athanasia Gkogkou, Vasileia Masoura and Despina Hatzidimitriou for the helpful conversations and advise.

\vspace*{0.5cm}
A special thank you goes to Evgenia Koutsoumpou and John Kallimanis for their daily support, help and friendship during this project, as well as to Despina Karavola, Argyris Loules, Stamatis Stathopoulos, Marios Nikolaidis, Barbara Kotsiourou, Vasilis Mpisketzis and Haris Tsakonas for being the most excellent colleagues and friends. Thank you to Sofia Zarbouti who cheered for and supported everyone of us. You all made it so much more special!  

Thank you to all the students and to all the people working at the University of Athens. 


\vspace*{0.5cm}

This work has been inspired by the research of Markos Polkas (<<The stellar mass content and star formation rate of radio galaxies as a function of redshift. Consequences for the growth of massive galaxies>>) on the project <<Do massive winds induced by black-hole jets alter galaxy evolution? Evidence from galaxies in the Atacama Large Millimeter Array (ALMA) Radio-source Catalog>> (PI: Kalliopi Dasyra, National Observatory of Athens).

In this work, the main data analysis tool has been the \code{UltraNest} software, (developed by Johannes Buchner) presented in 2021, Journal of Open Source Software, 6, 60, 3001 (DOI: \url{10.21105/joss.03001}).

This work has made use of the ARC data selection (original work by Audibert, Dasyra, Papachristou, Fernández-Ontiveros, Ruffa, Bisigello, Combes, Salomé and Gruppioni) that was published in 2022, Astronomy \& Astrophysics, 668, A67, 21 (DOI: \url{10.1051/0004-6361/202243666}).


I am grateful to the programmers who build open-source and publicly
available python packages.